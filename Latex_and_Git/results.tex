\section{Results}

A repeated measures 2 x 3 between-within ANOVA was conducted to assess changes in FAD behaviors over time and determine if the PNF for FAD condition demonstrates the most significant effects (see Table~\ref{tab:study}). Additionally, a mediation analysis was be conducted to examine whether FAD norms mediate the relationship between intervention type and FAD behaviors.

An a priori power analysis was conducted to determine the required sample size to detect a small effect size for a repeated-measures ANOVA (within-between interaction) using G*Power 3.1. It was decided to use a small effect size (f = 0.10) to be conservative, as no previous PNF interventions for FAD could be used as a reference. Past research indicated a small-moderate effect size of PNF interventions for alcohol use and problems (Dotson et al., 2015; Young \& Neighbors, 2019). The analysis was based on an alpha level of 0.05 and a desired power of 0.80. There were two groups (i.e., PNF-FAD, and control) and three measurement points (i.e., baseline, two-week and four-week follow-ups). The analysis indicated a minimum sample size of 164 participants per condition with these parameters to detect a statistically significant effect. Additionally, one research article suggests that a sample size of at least 562 participants is required to detect small effects in a partial mediation model with 0.80 power (Fritz \& Mackinnon, 2007). The goal was to recruit 650 participants to account for potential dropout rates.

\begin{table}[h!]
\centering
\caption{\label{tab:study}Engagement in FAD behaviors in the experimental and control groups. 0 = Never, 6 = Always.}
\begin{tabular}{l l c c c}
 \toprule
                          &              & \multicolumn{3}{c}{Study}\\
  \cmidrule{3-5}
                          &              & {Baseline}           & {Time 2}           & {Time 3}\\
  \cmidrule{3-5}
  \multirow{ 2}{*}{Group} & Experimental &     1.25    &      1.06     &      1.01     \\
                          & Control      &     1.23      &      1.67     &      1.23   \\
  \bottomrule
 \end{tabular}
\end{table}


\begin{figure}[ht!]
\centering
\begin{tikzpicture}[
  mynode/.style={draw, align=center, minimum width=2.5cm, inner sep=6pt}
]

% Nodes
\node[mynode] (M) {FAD Norms};
\node[mynode, below left=of M] (A) {Intervention};
\node[mynode, below right=of M] (B) {FAD Behaviors};

% Paths (only indirect mediation)
\draw[-latex, thick] (A.north) -- (M.south west);
\draw[-latex, thick] (M.south east) -- (B.north);
\draw[-latex, thick] (A.north) -- (B.north);

\end{tikzpicture}
\caption{\label{fig:TikZmodel}Mediation model of intervention effects on changes in FAD behaviors through changes in perceived FAD norms.}
\end{figure}