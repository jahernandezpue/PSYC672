\section{Results}
Here is where you will describe your significant results. Typically, you will place a number of figures in the results section to illustrate your findings. For example, I tried to use TiKZ to create a simple illustration of the factor structure of general intelligence, but I don't think it's very good (see Figure~\ref{fig:TikZmodel}).

\begin{figure}[ht!]
\centering
\begin{tikzpicture}[scale=.5,transform shape,auto,node distance=.9cm,
latent/.style={circle,draw,very thick,inner sep=0pt,minimum size=25mm,align=center},
manifest/.style={rectangle,draw,very thick,inner sep=0pt,minimum width=30mm,minimum height=10mm},
paths/.style={->, ultra thick, >=stealth'}]
\node [manifest] (SI) at (0,0) {Similarities};
\node [manifest] (VO) [below=of SI] {Vocabulary};
\node [manifest] (CO) [below=of VO] {Comprehension};
\node [manifest] (IN) [below=of CO] {Information};
\node [manifest] (WR) [below=of IN] {Word Reasoning};
\node [manifest] (BD) [below=of WR] {Block Design};
\node [manifest] (PS) [below=of BD] {Picture Concepts};
\node [latent] (g) [left=2.5cm of IN] {\emph{g}};
\node [latent] (Gc) [right=4.5cm of CO] {Verbal\\ Comprehension};
\node [latent] (Gv) [right=4.5cm of BD] {Visual\\ Spatial};
\foreach \all in {SI, VO, CO, IN, WR, BD, PS}
{
\draw [paths] (g.east) to node { } (\all.west);
}
\foreach \vc in {SI, VO, CO, IN, WR}
\draw [paths] (Gc.west) to node {} (\vc.east);
\foreach \vs in {BD, PS}
\draw [paths] (Gv.west) to node {} (\vs.east);
\draw [paths, <->] (Gc) edge[bend left] node [left] {} (Gv);
\end{tikzpicture}
\caption{\label{fig:TikZmodel} Here is a sample of a TikZ model.}
\end{figure}

In other cases, you might have an image file you want to bring into your document.  That is pretty easy to do as well as you can see in Figure \ref{fig:gearhead}.

\begin{figure}[h!] 
\centering
\includegraphics[width=0.3\textwidth]{frog.jpeg}
\caption{\label{fig:gearhead}This is a another figure caption.}
\end{figure}

Sometimes You may also want to display some tabular data in order to illustrate patterns in your data. Fortunately, You can do this right in \LaTeX.

\begin{table}[h!]
\centering
\caption{\label{tab:RTmeans}Reaction times over training in the experimental and control groups.}
\begin{tabular}{l l c c c}
 \toprule
                          &              & \multicolumn{3}{c}{Week}\\
  \cmidrule{3-5}
                          &              & {One}           & {Two}           & {Three}\\
  \cmidrule{3-5}
  \multirow{ 2}{*}{Group} & Experimental &     1008.435    &      986.76     &      859.1     \\
                          & Control      &     996.23      &      901.67     &      1002.23   \\
  \bottomrule
 \end{tabular}
\end{table}
