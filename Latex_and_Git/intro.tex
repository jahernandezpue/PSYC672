\section{Introduction}

Food and alcohol disturbance (FAD) involves disordered eating behaviors before, during, and/or after alcohol consumption used to compensate for alcohol-related calories (FAD-calories) or enhance alcohol’s effects (FAD-intoxication; \citep{Choquette2018}). Research on FAD has focused on prevalence rates, correlates, and the outcomes associated with underlying motives. Although FAD endorsement rates vary, one study suggests that in the past month, 67.1\% of college students endorsed engaging in FAD for either motive at least once. These risky drinking habits can have serious negative consequences, such as poor academic performance, legal repercussions, and the development of alcohol use disorder. High rates of FAD among students and limited research on interventions call for urgent development and evaluation of strategies to reduce FAD in high-risk college settings. Current interventions for FAD are limited and fail to address the full range of FAD motives and behaviors, highlighting a significant gap in the literature.

Recent research indicates that college students overestimate both the prevalence and acceptance of FAD among peers, suggesting that college students may benefit from addressing these normative misperceptions. Studies in alcohol use support this approach, showing that personalized normative feedback (PNF) interventions can effectively reduce alcohol use and associated harms and shift norms \citep{Lewis2007} \citep{Young2019}. The potential to adapt this method for addressing other harmful behavioral patterns, such as FAD, remains unexplored.

This study aimed to develop a PNF intervention to reduce FAD behaviors among college students. For this study, it is hypothesized that participants in the PNF-FAD condition will exhibit a greater reduction in FAD behaviors than the control group. Additionally, it is hypothesized that those with shifts in norms will have the greatest reductions in FAD behaviors and related alcohol outcomes (see Figure~\ref{fig:TikZmodel}).
