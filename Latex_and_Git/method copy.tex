\section{Method}

College students (N=X) were recruited from research participant pools or classes within the Psychology Department at William \& Mary in the Fall 2025 semester for an online survey and brief intervention focusing on personal health habits. The survey lasted about one hour, and participants were compensated with research credit. Participants completed a baseline survey, and two additional surveys spaced out two weeks apart (i.e., T1 [baseline], T2 [2-week follow-up post baseline survey], T3 [1-month post baseline follow-up survey]). The study protocol received IRB approval via university specific IRB.

\subsection{Participants}

Analytic sample was limited to students who reported consuming alcohol in the past 30 days, completed alcohol consumption and alcohol-related problems measures, and correctly responded to at least three of the five attention checks in the online survey.

\subsection{Materials and Procedure}

Participants received a personalized normative feedback (PNF) intervention during study 1. This intervention was similar to Young \& Neighbors (2019) PNF intervention for alcohol use. Participants were asked to report on their (1) personal frequency of FAD behavior, (2) perception of frequency of FAD behavior among peers, (3) perception of peers approval or disapproval of engaging in FAD behaviors, and (4) personal approval or disapproval of engaging in FAD behaviors. This information was used to compare to the actual FAD behaviors and norms (descriptive and injunctive) of typical same-sex peers at their university. This feedback was presented as text and graphs (see Figure~\ref{fig:gearhead}). The information for comparison with same-sex peers was gathered from a larger project that collected information on FAD behaviors and FAD norms among W\&M students (Project SNAP3). For the control group, participants received the same information related to their sleep quality (Project Koala). For the follow-up surveys, participants completed the same series of measures as in the baseline questionnaire.

\begin{figure}[h!] 
\centering
\includegraphics[width=0.7\textwidth]{FAD1-Female.png}
\caption{\label{fig:gearhead}Example of FAD feedback in graph.}
\end{figure}